\documentclass{report}
\title{Laboratorio del corso direti logiche 2021/2022}

\begin{document}


Documento scritto tra il 26/11/2021 e il 28/11/2021 da tutti i componenti del gruppo:

Davide Frageri
Federico Natali
Matteo Sabella
(Gabriele Fasolli)
(Alessandra ...)

\section*{Obiettivi e specifica}

L'esperienza di laboratorio si poneva l'obiettivo di realizzare una calcolatrice costitutita dai seguenti elementi:

\begin{itemize}
\item Debouncer :
blocco che permette di eliminare i rumori dovuti all'utilizzo del pulsante.
\item Alu:
unità aritmetico logica che realizza la computazione  numerica o l'eventuale reset.
\item Accumulator:
unità che regolava l'uscita della risposta in base al clock (in quanto il nostro sarebbe dovuto essere un circuito asincrono) e in base ad un segnale di enable.
\end{itemize}

la rappresentazione dei numeri sarebbe avvenuta su schermo oled nel nostro caso tramite i driver forniti sulla piattaforma dal docente e i numeri sarebbero stati in binario visto che per inserirli disponevamo di soli interruttori.
I numeri erano per specifica signed da 32 bit.

Le operazioni da implementare sarebbero state addizione, sottrazione e moltiplicazione, selezionabili tramite pulsanti e l'operazione selezionata doveva essere effettuata tra il numero inserito dall'utente e il risultato dell'ultima operazione eseguita, a meno che non fosse la prima oppure l0utente non avesse deciso di eseguire un reset.

La scheda da noi utilizzata era la zedboard.


\section*{Schema a blocchi}
\section*{Componenti}
\section*{Risultati}


\end{document}